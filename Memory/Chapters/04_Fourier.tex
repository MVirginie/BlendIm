Avec cette seconde méthode nous allons résoudre l'équation de Poisson à l'aide de la transformée de Fourier. Avant de formuler la résolution de ce problème. Rappelons la définition des opérateurs dont nous aurons besoin dans la suite
\subsubsection{Rappel et définitions des opérateurs}
\paragraph{Transformée de Fourier}
Soit S une fonction, sa transformée de Fourier peut s'écrire de la façon suivante : 
\begin{equation}
\begin{aligned}
\widehat{S}(x,y) = \sum_{k = 0}^{M-1} \sum_{l = 0}^{N-1} S(k,l) e^{-2\pi i\left(\frac{k\times x}{M}+\frac{l\times y}{N}\right)}
\end{aligned}
\end{equation}
Enfin afin de retrouver la fonction initiale nous aurons besoin de la transformée de Fourier inverse : 
\begin{equation}
\begin{aligned}
S(k,l) = \frac{1}{MN} \sum_{x = 0}^{M-1} \sum_{y = 0}^{N-1} \hat{S}(x,y) e^{2\pi i \left(\frac{xk}{M}+\frac{yl}{N}\right)}
\end{aligned}
\end{equation}
\paragraph{Gradient}
Nous pouvons calculer le gradient d'une fonction dans le domaine de Fourier. Nous considérons toujours S la fonction que nous souhaitons étudier. 
\begin{equation}
\begin{aligned}
\widehat{\nabla (S)}=
\begin{pmatrix}
\widehat{\frac{\partial S}{\partial k}}\\
\widehat{\frac{\partial S}{\partial l}}
\end{pmatrix}
\end{aligned}
\end{equation}
En dérivant l'expression ci-dessus par rapport à la première variable : 
\begin{equation}
\begin{aligned}
\frac{\partial S}{\partial k} &= \sum_{x = 0}^{M-1} \sum_{y = 0}^{N-1} S(x,y) e^{2\pi i\left(\frac{k\times x}{M}+\frac{l\times y}{N}\right)}\left(\frac{2\pi i x}{M}\right)\\
& = \left(\frac{2\pi i x}{M}\right)\widehat{S(k,l)}\\
\widehat{\frac{\partial S}{\partial k}} &= \left(\frac{2\pi i x}{M}\right)\widehat{S(k,l)}
\end{aligned}
\end{equation}
Le calcul est similaire pour $\widehat{\frac{\partial S}{\partial l}}$.\\
On a donc : 
\begin{equation}
\begin{aligned}
\widehat{\frac{\partial S}{\partial k}} = \left(\frac{2\pi i}{M}x\right) \widehat{S}\\
\widehat{\frac{\partial S}{\partial l}} = \left(\frac{2\pi i}{N}y\right) \widehat{S}\\
\end{aligned}
\end{equation}

\paragraph{Laplacien}
\begin{equation}
\begin{aligned}
\frac{\partial^2 S}{\partial k ^2} & = \sum_{x = 0}^{M-1} \sum_{y = 0}^{N-1} S(x,y) e^{2\pi i\left(\frac{k\times x}{M}+\frac{l\times y}{N}\right)}\left(\frac{2\pi i x}{M}\right)^2\\
& = \left(\frac{2\pi i x}{M}\right)^2 \widehat{S(k,l)}\\
\widehat{\frac{\partial^2 S}{\partial k^2}} &= \left(\frac{2\pi i x}{M}\right)^2\widehat{S(k,l)}
\end{aligned}
\end{equation}
On a donc : 
$\widehat{\Delta S} = \widehat{\frac{\partial^2 S}{\partial k^2}}+ \widehat{\frac{\partial^2 S}{\partial l^2}}$.
\begin{equation}
\widehat{\Delta S} = \left(\frac{2\pi i x}{M}\right)^2 \widehat{S}+\left(\frac{2\pi i y}{N}\right)^2 \widehat{S}
\end{equation}

\subsubsection{Résolution de la méthode Fourier}
La résolution dans le domaine de Fourier, nécessite quelques changements.Cette méthode ne fonctionnant que sur un domaine rectangulaire, nous devons donc modifier le domaine $\Omega$.\\
Rappelons que nous voulions résoudre le problème suivant :
\begin{equation}
\left\{
\begin{aligned}
\Delta I(x,y) = \Delta S(x,y) \ si \ (x,y) \in \Omega\\
I(x,y) = T(x,y) \ si \ (x,y) \notin \Omega
\end{aligned}
\right.
\end{equation}

Afin d'obtenir un domaine régulier considérons R, le domaine de l'image entière. Afin de pouvoir utiliser Fourier, il faut que nous puissions travailler symétriquement en même temps sur les deux images.\\
Nous voulons donc que $\nabla I$ soit très proche de $\nabla S$ dans $\Omega$ mais aussi que $\nabla I$ soit très proche de $\nabla T$ en dehors de $\Omega$. En notant V : 
\begin{equation}
V = 
\left\{
\begin{aligned}
\nabla S \in \Omega\\
\nabla T \notin \Omega
\end{aligned}
\right.
\end{equation}
Nous devons donc résoudre l'équation suivante : 
\begin{center}
$ \Delta I = div(V)$
\end{center}
Afin de pouvoir résoudre cette équation il faut imposer des conditions sur le bord. En appliquant l'effet miroir à l'image initiale alors, l'on obtient un signal symétrique  et l'on peut appliquer la transformée de Fourier, pour résoudre le problème. 

Ainsi en calculant le laplacien dans le domaine de Fourier, on obtient  : $\widehat{\Delta I} = \widehat{div(V)}$

\begin{equation}
\begin{aligned}
\left(\frac{2\pi i x}{M}\right)^2 \widehat{I}+\left(\frac{2\pi i y}{N}\right)^2 \widehat{I} & = \left(\frac{2\pi i x}{M}\right) \widehat{V_1}+\left(\frac{2\pi i y}{N}\right) \widehat{V_2}\\
\left(\left(\frac{2\pi i x}{M}\right)^2+\left(\frac{2\pi i y}{N}\right)^2\right) \widehat{I} & = \left(\frac{2\pi i x}{M}\right) \widehat{V_1}+\left(\frac{2\pi i y}{N}\right) \widehat{V_2}\\
\widehat{I} = \frac{\left(\frac{2\pi i x}{M}\right) \widehat{V_1}+\left(\frac{2\pi i y}{N}\right) \widehat{V_2}}{\left(\left(\frac{2\pi i x}{M}\right)^2+\left(\frac{2\pi i y}{N}\right)^2\right)}
\end{aligned}
\end{equation}
Afin de retrouver I, il suffit d'appliquer la transformée inverse, à l'équation ci-dessus.