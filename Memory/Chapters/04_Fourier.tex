Avec cette seconde méthode nous allons résoudre l'équation de Poisson à l'aide de la transformée de Fourier. Avant de formuler la résolution de ce problème. Rappelons la définition des opérateurs dont nous aurons besoin dans la suite
\subsubsection{Rappel et définitions des opérateurs}
\paragraph{Transformée de Fourier}
Soit S une fonction, sa transformée de Fourier peut s'écrire de la façon suivante : 
\begin{equation}
\begin{aligned}
\widehat{S}(x,y) = \sum_{k = 0}^{W-1} \sum_{l = 0}^{H-1} S(k,l) e^{-2\pi i\left(\frac{k\times x}{W}+\frac{l\times y}{H}\right)}
\end{aligned}
\end{equation}
Enfin afin de retrouver la fonction initiale nous aurons besoin de la transformée de Fourier inverse : 
\begin{equation}
\begin{aligned}
S(k,l) = \frac{1}{WH} \sum_{x = 0}^{W-1} \sum_{y = 0}^{H-1} \hat{S}(x,y) e^{2\pi i \left(\frac{xk}{W}+\frac{yl}{H}\right)}
\end{aligned}
\end{equation}
\paragraph{Gradient}
Nous pouvons calculer le gradient d'une fonction dans le domaine de Fourier. Nous considérons toujours S la fonction que nous souhaitons étudier. 
\begin{equation}
\begin{aligned}
\widehat{\nabla (S)}=
\begin{pmatrix}
\widehat{\frac{\partial I}{\partial k}}\\
\widehat{\frac{\partial I}{\partial l}}
\end{pmatrix}
\end{aligned}
\end{equation}
En dérivant l'expression ci-dessus par rapport à la première variable : 
\begin{equation}
\begin{aligned}
\frac{\partial S}{\partial k} &= \sum_{x = 0}^{W-1} \sum_{y = 0}^{H-1} S(x,y) e^{2\pi i\left(\frac{k\times x}{W}+\frac{l\times y}{H}\right)}\left(\frac{2\pi i x}{W}\right)\\
& = \left(\frac{2\pi i x}{W}\right)S(k,l)\\
\widehat{\frac{\partial S}{\partial k}} &= \left(\frac{2\pi i x}{W}\right)\widehat{S(k,l)}
\end{aligned}
\end{equation}
Le calcul est similaire pour $\widehat{\frac{\partial S}{\partial l}}$.\\
On a donc : 
\begin{equation}
\begin{aligned}
\widehat{\frac{\partial S}{\partial k}} = \left(\frac{2\pi i}{W}x\right) \widehat{S}\\
\widehat{\frac{\partial S}{\partial l}} = \left(\frac{2\pi i}{H}y\right) \widehat{S}\\
\end{aligned}
\end{equation}

\paragraph{Laplacien}
\begin{equation}
\begin{aligned}
\frac{\partial^2 S}{\partial k ^2} & = \sum_{x = 0}^{W-1} \sum_{y = 0}^{H-1} S(x,y) e^{2\pi i\left(\frac{k\times x}{W}+\frac{l\times y}{H}\right)}\left(\frac{2\pi i x}{W}\right)^2\\
& = \left(\frac{2\pi i x}{W}\right)^2 S(k,l)\\
\widehat{\frac{\partial^2 S}{\partial k^2}} &= \left(\frac{2\pi i x}{W}\right)^2\widehat{S(k,l)}
\end{aligned}
\end{equation}
On a donc : 
$\widehat{\Delta S} = \widehat{\frac{\partial^2 S}{\partial k^2}}+ \widehat{\frac{\partial^2 S}{\partial l}^2}$.
\begin{equation}
\widehat{\Delta S} = \left(\frac{2\pi i x}{W}\right)^2 \widehat{S}+\left(\frac{2\pi i y}{H}\right)^2 \widehat{S}
\end{equation}

\subsubsection{Résolution de la méthode Fourier}
Rappelons que nous devons résoudre l'équation : 

Ainsi en calculant le laplacien dans le domaine de Fourier, on obtient  : $\widehat{\Delta I} = \widehat{\Delta S}$
\begin{equation}
\begin{aligned}
\left(\frac{2\pi i x}{W}\right)^2 \widehat{I}+\left(\frac{2\pi i y}{H}\right)^2 \widehat{I} & = \left(\frac{2\pi i x}{W}\right) \widehat{S}+\left(\frac{2\pi i y}{H}\right) \widehat{S}\\
\left(\left(\frac{2\pi i x}{W}\right)^2+\left(\frac{2\pi i y}{H}\right)^2\right) \widehat{I} & = \left(\frac{2\pi i x}{W}\right) \widehat{S}+\left(\frac{2\pi i y}{H}\right) \widehat{S}\\
\widehat{I} = \frac{\left(\frac{2\pi i x}{W}\right) \widehat{S}+\left(\frac{2\pi i y}{H}\right) \widehat{S}}{\left(\left(\frac{2\pi i x}{W}\right)^2+\left(\frac{2\pi i y}{H}\right)^2\right)}
\end{aligned}
\end{equation}
Afin de retrouver I, il suffit d'appliquer la transformée inverse, à l'équation ci-dessus.