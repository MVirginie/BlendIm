Nous verrons dans un premier temps, comment résoudre cette équation à l'aide de discrétisation et de différences finies. Puis nous résoudrons celle-ci à l'aide des transformées de Fourier. 
Rappelons que nous souhaitons résoudre l'équation de Poisson suivante : 
\begin{center}

\begin{equation*}
    \left \{
    \begin{aligned}
    \Delta I = \Delta S \ sur \ \Omega\\
    I = T \ sur \ \partial \Omega
    \end{aligned}
    \right.
\end{equation*}
\end{center}
%Nous cherchons ici à déterminer $\Delta I$ que nous ne connaissons pas. Au contraire, $\Delta S $ est facile à calculer, car nous connaissons S, et les valeurs de ses pixels. 
%\newline
%Rappelons que le Laplacien d'une fonction I(x,y) s'écrit de cette façon : 
%\begin{center}
%\begin{equation*}
%    \Delta I(x,y)  = \frac{\partial^2 I}{\partial x^2}+ \frac{\partial ^2 I}{\partial y^2}
%\end{equation*}
%\end{center}
