Afin d'aller plus loin, nous avons décidé d'appliquer nos précédents algorithmes sur une vidéo. \\
Nous considérons qu'une vidéo peut-être vue comme une succession d'images. En appliquant les algorithmes sur chacune de ces images nous pourrons donc incruster un objet dans une vidéo. \\
Nous avons commencé par essayer d'incruster une image fixe dans une vidéo. 
L'image S reste donc comme précédemment l'image initiale à coller, mais l'image T, elle, devient une vecteur d'images. En appliquant les algorithmes implémentés sur chacune de ces images,considérons n, le nombres d'images issues de la vidéo, il y a donc n résultats différents et il est donc possible de reconstituer une nouvelle vidéo.  Le résultat est affiché une fois toutes les images calculées. \\ Nous ne pouvons malheureusement pas appliquer nos algorithmes "en temps réel", en effet les méthodes utilisées mettant à peu près une seconde pour recalculer les images, nous n'obtenons donc pas une vidéo mais un diaporama d'images. \\
Dans un second temps, nous voudrions incruster une vidéo sur une image fixe, c'est en quelque sorte le principe de gif. L'image S serait alors un vecteurs d'image, l'image T, elle, étant une simple image. \\ Dans ce cas nous ne pouvons pas sélectionner une région à coller. L'objet de la vidéo étant en mouvement dans celle-ci, il faudrait donc sélectionner celui-ci dans chacune des images extraites de la vidéo. Ou encore faire de la détection d'objet, de la détection de mouvement, un tracking "suivi". Nous pouvons néanmoins essayer d'incruster toute la vidéo dans l'image T.  C'est-à-dire coller chaque image entière de la vidéo dans l'image T.\\
Enfin nous pourrions peut-être essayer d'incruster une vidéo dans une autre. \\
Les temps de calcul sont relativement longs, puisque chaque image doit être recalculée. Cependant nous pouvons sans doute améliorer ceux-ci à l'aide de "Threads". \\
Nous mettons dans une annexe la vidéo initiale, l'image initiale et enfin les résultats obtenus, pour l'incrustation d'une image dans une vidéo. 
Nous avons extrait 100 images de la vidéo initiale, soit 20 images par seconde. Ces images sont de taille : $1364 \times 768 \times 3$.  L'image S quand a elle fait : $220 \times 154 \times 3$.