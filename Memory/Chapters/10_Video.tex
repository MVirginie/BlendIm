Afin d'aller plus loin, nous avons décidé d'appliquer nos précédents algorithmes sur une vidéo. \\
Nous considérerons une vidéo comme une succession d'images. En appliquant les algorithmes sur chacune de ces images nous pourrons donc incruster un objet dans une vidéo. \\
Nous avons commencé par essayer d'incruster une image fixe dans une vidéo. 
L'image S reste donc comme précédemment l'image initiale à coller, mais l'image T, elle, devient une vecteur d'images. En appliquant les algorithmes implémentés sur chacune de ces images,nous pourrons obtenir le résultat souhaité. En effet, considérons n le nombres d'images issues de la vidéo, en utilisant nos algorithmes sur chacune d'elles, il y a donc n résultats différents et il est donc possible de reconstituer une nouvelle vidéo à partir des images obtenues.  Le résultat est affiché une fois toutes les images calculées. \\ 
Nous ne pouvons malheureusement pas appliquer nos algorithmes "en temps réel" sur de "grandes" images, en effet nos algorithmes mettant plus ou moins de temps selon les images choisies, le résultat obtenu en temps réel ne serait donc pas une vidéo mais un diaporama d'images. Nous optimiserons chacun de nos algorithmes afin de pouvoir essayer ces schémas en temps réel. \\
Vous trouverez en annexe, les vidéos obtenues avec les méthodes des différences finies et de Fourier. 
Nous avons extrait 100 images de la vidéo initiale, soit 20 images par seconde. Ces images sont de taille : $1364 \times 768 \times 3$.  L'image S quand à elle,fait : $220 \times 154 \times 3$.\\
Nous voulions essayer d'aller un peu plus loin en incrustant une vidéo sur une image fixe. Dans ce cas, l'image S serait un vecteur d'image, l'image T, elle, étant une simple image. \\ Malheureusement, nous ne pouvons pas sélectionner une région à coller. L'objet de la vidéo étant en mouvement dans celle-ci, il faudrait donc sélectionner celui-ci dans chacune des images extraites de la vidéo. Ou encore faire de la détection d'objet, de la détection de mouvement, un "suivi" de l'objet afin de modifier la sélection à chaque instant. Nous pouvons néanmoins essayer d'incruster la vidéo entière dans l'image T.  C'est-à-dire coller chaque image de la vidéo, entièrement dans l'image T. Malheureusement il faut donc que l'objet à sélectionner soit sur un fond plutôt lisse afin qu'il s'incruste parfaitement et que les objets autour ne se retrouve pas eux aussi dans le résultat final. \\
En arrivant à incruster une vidéo sur une image nous pourrions facilement essayer d'incruster une vidéo dans une autre. \\
Afin d'accélérer la vitesse des algorithmes, nous proposons d'utiliser le calcul parallèle. Nous pourrions alors "construire" plusieurs images en même temps et ainsi améliorer considérablement la vitesse de nos algorithmes .\\
