Cette partie sera consacrée à la résolution du problème à l'aide des différences finies. Pour trouver la solution il faut donc discrétiser les laplaciens des images, puis résoudre un système.
 \subsubsection{Notations}
\textbf{Le gradient: }\\
Le gradient est un vecteur composé des dérivées partielles d'une fonction. Soit la fonction, f(x,y), on note le gradient de f : 
\begin{equation*}
\begin{aligned}
\nabla f = \begin{pmatrix}
\frac{\partial f}{\partial x}\\
\frac{\partial f}{\partial y}
\end{pmatrix}
\end{aligned}
\end{equation*}
\textbf{Le Laplacien}\\
On note le Laplacien : $\Delta$, et :$\Delta = div(\nabla f)$.
\begin{center}
\begin{equation*}
    \Delta f  = \frac{\partial^2 f}{\partial x^2}+ \frac{\partial ^2 f}{\partial y^2}
\end{equation*}
\end{center}

\subsection{Méthode des différences finies }
Le problème à résoudre est le suivant : 
\begin{center}

\begin{equation*}
    \left \{
    \begin{aligned}
    \Delta I = \Delta S \ sur \ \Omega\\
    I = T \ sur \ \partial \Omega
    \end{aligned}
    \right.
\end{equation*}
\end{center}
Dans un premier temps, discrétisons le Laplacien de I. Le laplacien étant la somme des dérivées partielles secondes : $\frac{\partial^2 I}{\partial x^2}$ et  $\frac{\partial^2 I}{\partial y^2}$, sa discrétisation commence par une discrétisation de celles-ci. En utilisant les formules de Taylor-Young à l'ordre 2 suivantes :
\begin{equation*}
\begin{aligned}
    I(x+h,y) = I(x,y)+h\times \frac{\partial I(x,y)}{\partial x}+ \frac{h^2}{2} \times \frac{\partial ^2 I(x,y)}{\partial x^2} + o(h^3) \\
    I(x-h,y) =I(x,y)- h\times  \frac{\partial I(x,y)}{\partial x}+ \frac{h^2}{2} \times \frac{\partial^2 I(x,y)}{\partial x^2} + o(h^3)
\end{aligned}
\end{equation*}
Il est donc facile de voir que la somme de ces deux équations permet d'obtenir une discrétisation de la dérivée seconde : $\frac{\partial ^2 I(x,y)}{\partial x^2}$:  
\begin{equation*}
    \frac{\partial ^2 I(x,y)}{\partial x^2} =\frac{1}{h^2}\left( I(x+h,y) + I(x-h,y) - 2\times I(x,y)\right)
\end{equation*}
La somme des discrétisations des dérivées secondes : $\frac{\partial ^2 I(x,y)}{\partial x^2}$ et $\frac{\partial ^2 I(x,y)}{\partial y^2}$, permet d'obtenir une discrétisation possible du Laplacien de I : $\Delta I$.
\begin{equation*}
    \Delta I(x,y) =  \frac{I(x+h,y) + I(x-h,y) - 2\times I(x,y)}{h^2}  + \frac{I(x,y+k) + I(x,y-k) - 2\times I(x,y)}{k^2} \\
\end{equation*}

Les pas d'espaces h et k étant égaux à 1, nous pouvons écrire une discrétisation du Laplacien  :
\begin{equation*}
     \Delta I(x,y) =  I(x+1,y) + I(x-1,y)+ I(x,y+1) + I(x,y-1) - 4\times I(x,y)  \\
\end{equation*}

\paragraph{Application à une image }
Afin d'obtenir le Laplacien du pixel I(i,j), il est donc nécessaire d'avoir la connaissance de ses pixels voisins que nous nommerons par la suite U(p), D(own), L(eft), R(ight) pour les pixels I(i-1,j), I(i+1,j), I(i,j-1), I(i,j+1). 

\begin{figure}
\centering
    \includegraphics[scale = 0.8]{Images/Laplacian.png}
    \caption{Laplacien du pixel}
\end{figure}
Afin de trouver la solution au problème, il faut donc appliquer cette discrétisation à chaque pixel de la partie de l'image à modifier($\Omega \cap \partial \Omega$). Chaque pixel faisant intervenir ses voisins, la résolution du problème passe donc par la résolution d'un système.
En notant
\begin{center}
 $g(x,y) = S(x+1,y) + S(x-1,y)+ S(x,y+1) + S(x,y-1) - 4\times S(x,y)$\\
 \end{center}
Résoudre $\Delta I(x,y) = \Delta S(x,y)$ sur $\Omega$ est équivalent à résoudre :\\
\begin{center}
\begin{equation*}
    \left \{
    \begin{aligned}
    I(i+1,j) + I(i-1,j)+ I(i,j+1) + I(i, j-1) - 4\times 			I(i,j)= g(i,j)\\ pour (i,j)\in \Omega \\
    I(i,j) = T(i,j) \ pour \ (i,j) \in \partial \Omega
    \end{aligned}
    \right.
\end{equation*}
\end{center}
La résolution de ce système de taille $M\times N $, permettra de trouver la nouvelle image I, et donc de résoudre numériquement l'équation de Poisson avec conditions aux bords de Dirichlet. Voici, le système obtenu : 
\begin{equation}
\left\{
\begin{aligned}
I(1,1) = T(1,1)\\
I(3,2)+I(1,2)+ I(2,3)+I(2,1)-4I(2,2) =g(2,2) \\
I(3,3)+I(1,3)+ I(2,4)+I(2,2)-4I(2,3) =g(2,3)             \\
... \\
I(M,N-1)+I(M-2,N-1)+ I(M-1,N)+I(M-1,N-2)-4I(M-1,N-1) =g(M-1,N-1)\\
I(M, N) = T(M, N)
\end{aligned}
\right.
\end{equation}
\subsubsection{Résolution du système}
Afin de résoudre ce système, il est plus facile de l'écrire sous forme matricielle. Il faut maintenant trouver la solution du problème suivant : 
\begin{center}
 AI = b 
\end{center}
Si la matrice est inversible, alors la solution est évidente, et elle vaut : 
\begin{center}
$I = A^{-1}\times b$
\end{center} 
Avec : 
\begin{itemize}
\item A, une matrice carrée de taille ($M\times N$, $M\times N$)
\item I, un vecteur colonne de taille ($M\times N$,1)
\item b, un vecteur colonne de taille ($M\times N$,1)
\end{itemize}
Voici donc à quoi ressemble le système que nous souhaitons résoudre :
La matrice A, est une matrice par bloc. Ainsi les blocs de la matrice correspondant aux pixels à l'intérieur du domaine seront remplis de la manière  suivante : 
\begin{center}
\begin{equation}
\left.
\begin{aligned}
\begin{pmatrix}
	-4 & 1 & 0 & ...& 0 & 1 & 0...&0& ... & 0\\
	1 & -4 & 1 & 0 & ... & 0 &1 &0&....&0\\
	0 & 1 & -4 & 1 & 0&... &0 &1 &0&...\\
	&...\\
	1 & 0 &... &1 &-4 &1 &0...& 0& 1 & 0\\
\end{pmatrix}
\end{aligned}
\right.
\end{equation}
\end{center}
Tandis que les blocs de la matrice, correspondant aux pixels situés sur le bords du domaine seront remplis de la façon suivante : 
\begin{center}
\begin{equation}
\left.
\begin{aligned}
\begin{pmatrix}
	1 & 0& 0 & ...& 0 & 0 & 0...&0& ... & 0\\
	0 & 1 & 0 & 0 & ... & 0 &0 &0&....&0\\
	0 & 0 & 1 & 0 & 0&... &0 &0 &0&...\\
	&...\\
	0 & 0 &... &0 &0 &0 &0...& 0& 0 & 1\\
\end{pmatrix}
\end{aligned}
\right.
\end{equation}
\end{center}
Le vecteur b, contient les laplaciens des pixels de l'image S.
Les valeurs des pixels de celle-ci, étant connus,il est facile de calculer son Laplacien en utilisant les discrétisations vues ci-dessus. Le vecteur b, n'est autre que la concaténation des colonnes du Laplacien de S.
\begin{center}
\begin{equation}
\left.
\begin{aligned}
\begin{pmatrix}
g(1,1)\\
g(2,1)\\
...\\
g(1,2)\\
g(2,2)\\
....\\
g(M, N)
\end{pmatrix}
\end{aligned}
\right.
\end{equation}
\end{center}
En inversant la matrice, la solution I obtenue est un vecteur : 
\begin{center}
\begin{equation}
\left.
\begin{aligned}
\begin{pmatrix}
I(1,1)\\
I(2,1)\\
...\\
I(1,2)\\
I(2,2)\\
....\\
I(M, N)
\end{pmatrix}
\end{aligned}
\right.
\end{equation}
\end{center}
Afin de trouver l'image finale, il est donc important de reconstruire une matrice de taille $M\times N$, à partir de ce vecteur. 

\subsection{Exemple}
Considérons l'image S ci-contre que nous souhaitons coller. 
\begin{figure}[!htb]
   \begin{minipage}{0.5\textwidth}
     \centering
     \includegraphics[width = 120pt]{Images/square.png}
     \caption{Images à coller}
      \end{minipage}\hfill
   \begin{minipage}{0.5\textwidth}
     \centering
\includegraphics[width = 120pt]{Images/pix.png}
\caption{Vue grille pixel}
      \end{minipage}\hfill
\end{figure}
Nous souhaitons ici coller les deux carrés rouge sur une image que nous nommerons T.  

\begin{figure}[!htb]
   \begin{minipage}{0.5\textwidth}
     \centering
     \includegraphics[width = 120pt]{Images/carre_selection.png}
\caption{Sélection à coller}
      \end{minipage}\hfill
   \begin{minipage}{0.5\textwidth}
     \centering
\includegraphics[width = 120pt]{Images/numerote.png}
\caption{Numérotation des pixels}
      \end{minipage}\hfill
\end{figure} 
\newpage

Ici, notons $\Omega$, l'intérieur de la zone encadrée par la ligne verte, et des $\partial \Omega$, les pixels appartenant à $Jaune \backslash \Omega$


\paragraph{Construction du système}
\begin{center}
\begin{equation}
\left\{
\begin{aligned}
I_{1,1} = T_{1,1}\\
...\\
I_{5,1} = T_{5,1}\\
I_{1,2} = T_{1,2}\\
\Delta I_{2,2} = \Delta S_{2,2}\\
\Delta I_{3,2} = \Delta S_{3,2}\\
\Delta I_{4,2} = \Delta S_{4,2}\\
...\\
\end{aligned}
\right.
\end{equation}
\end{center}
\newpage
Avec $\Delta I(i,j) = U+D+L+R-4I(i,j)$.\\
En écrivant ce système sous forme matricielle la matrice A est la matrice $25\times 25$ ci-dessous :  

\begin{figure}[!htb]
\includegraphics[scale=0.5]{Images/matrice.png}
\caption{Matrice du système}
\end{figure}
\begin{equation}
I = 
\begin{pmatrix}
I(1,1)\\
I(2,1)\\
...\\
I(5,1)\\
...\\
I(1,3)\\
...\\
I(5,3)\\
...\\
I(5,5)\\
\end{pmatrix}
b = 
\begin{pmatrix}
T(1,1)\\
...\\
\Delta S(2,2)\\
...\\
T(5,2)\\
\Delta S(1,3)\\
...\\
T(5,3)\\
\Delta S(2,4)\\
...\\
\Delta S(4,4)\\
T(5,4)\\
...\\
T(5,5)\\
\end{pmatrix}
\end{equation}
La solution de ce système existe bien. En effet,A est carrée, elle est de taille $\left(M\times N, M \times N\right)$. Elle est aussi, toujours remplie de la manière, 'par blocs'. Ses colonnes étant linéairement indépendantes, elle est donc inversible.\\
La solution I, s'écrit sous la forme
\begin{center}
$I = A^{-1}b$.
\end{center} 
Nous ajouterons dans la section suivante les résultats obtenus à l'aide de cette méthode. \\ Celle-ci fonctionne très bien mais le temps de calcul peut devenir très long. En effet, cette méthode demande une inversion matricielle, et donc un temps de calcul relativement long sur de "très" grands systèmes, donc de très grandes sélections. Nous allons maintenant voir une seconde méthode, plus rapide, en nous plaçant dans le domaine de Fourier.

