L'équation de Poisson avec conditions aux bords de Dirichlet permet donc l'insertion d'une image dans une autre. A l'inverse, nous pouvons utiliser cette équations pour effacer certaines parties d'une image, ou encore pour augmenter la luminosité d'une image. Cette méthode est plutôt efficace même si elle possède ses limites.En effet, la couleur de l'objet à coller est par conséquent bien modifiée, qui parfois rends le collage beaucoup moins naturel. De même une grande sélection, est visible sur un fond très détaillé. Plus l'image de fond 'T' possède des variations, contours, détails, plus le collage devient mauvais. \\
Nous avons précédemment vu qu'une sélection plutôt large pouvait masquer des objets importants de l'image T. Ce problème peut être en partie résolu par des systèmes de comparaisons de gradients ou de moyenne mais le résultat n'est pas pas fiable, surtout si l'image à coller possède peu de variations. Cependant nous pouvons essayer d'étendre la résolution de cette équation sur des vidéos, en voulant incruster une vidéo dans une autre. Par exemple, réussir à incruster un vol d'oiseaux dans une vidéo. Si le vol se déplace dans la vidéo il faudrait être capable de détecter les mouvements de l'oiseau, (à l'aide de carte de disparité) afin de déplacer la sélection en même temps que l'objet.
\cite{Image}
\cite{Douglas}
\cite{Gradient}
\cite{Opti}
\cite{Fourier}