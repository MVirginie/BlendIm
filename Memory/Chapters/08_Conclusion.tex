L'équation de Poisson avec conditions aux bords de Dirichlet permet donc l'insertion d'une image dans une autre. À l'inverse, nous pouvons utiliser cette équations pour effacer certaines parties d'une image ou encore pour augmenter la luminosité d'une image. Cette méthode est plutôt efficace même si elle possède ses limites. En effet, la couleur de l'objet à coller est par conséquent modifiée, qui parfois rend le collage beaucoup moins naturel. De même une grande sélection, est visible sur un fond très détaillé. Plus l'image de fond 'T' possède des variations, contours, détails, plus le collage devient mauvais, (effet de flou sur l'image T plus visible). \\
Nous avons précédemment vu qu'une sélection plutôt large pouvait masquer des objets importants de l'image T. Ce problème peut être en partie résolu par des systèmes de comparaisons de gradients ou de moyenne mais le résultat n'est pas fiable, surtout si l'image à coller possède peu de variations et au contraire l'image T énormément.\\ 
L'équation de Poisson est utilisé dans beaucoup d'autres améliorations, telles que l'amélioration du contraste, l'augmentation de la luminosité de l'image, ou encore le traitement de films.