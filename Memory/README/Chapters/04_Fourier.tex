Avec cette seconde méthode nous allons résoudre l'équation de Poisson à l'aide de la transformée de Fourier. Avant de formuler la résolution de ce problème. Rappelons la définition des opérateurs dont nous aurons besoin dans la suite
\subsubsection{Rappel et définitions des opérateurs}
\paragraph{Transformée de Fourier}
Soit I une fonction, sa transformée de Fourier peut s'écrire de la façon suivante : 
\begin{equation}
\begin{aligned}
\hat{I}(x,y) = \sum_{i = 0}^{W-1} \sum_{j = 0}{H-1} I(i,j) e^{-2\pi i\left(\frac{xi}{W}+\frac{yj}{H}\right)}
\end{aligned}
\end{equation}
\paragraph{Gradient}
Nous pouvons calculer le gradient d'une fonction dans le domaine de Fourier. Nous considérons toujours I la fonction que nous souhaitons étudier. 
\begin{equation}
\begin{aligned}
\nabla (\hat{I}) =
\begin{pmatrix}
\frac{\partial\hat{I}}{\partial i}\\
\frac{\partial \hat{I}}{\partial j}
\end{pmatrix}
\end{aligned}
\end{equation}
Avec 
\begin{equation}
\begin{aligned}
\frac{\partial \hat{I}}{\partial i} = \left(\frac{2\pi i}{W}x\right) \hat{I}\\
\frac{\partial \hat{I}}{\partial j} = \left(\frac{2\pi i}{H}y\right) \hat{I}\\
\end{aligned}
\end{equation}
Enfin afin de retrouver la fonction initiale nous aurons besoin de la transformée de Fourier inverse : 
\begin{equation}
\begin{aligned}
I(i,j) = \frac{1}{WH} \sum_{x = 0}^{W-1} \sum_{y = 0}^{H-1} \hat{I}(x,y) e^{2\pi i \left(\frac{xi}{W}+\frac{yj}{H}\right)}
\end{aligned}
\end{equation}

\subsubsection{Résolution de la méthode Fourier}